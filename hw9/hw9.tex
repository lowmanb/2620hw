\documentclass[12pt]{2620hw}
\hwnum{9}
\due{March 28th, 2014}

\begin{document}
\maketitle

\begin{enumerate}

\item [44)]
\begin{enumerate}
    \item Solving the total energy equation for $k_1$ 
	\[
		E = \frac{\hbar^2 k_1^2}{2m}
	\]
	\[
		k_1 = \frac{\sqrt{2mE}}{\hbar}
	\]
	For $k_2$, $E \Rightarrow E-(V_0) = E-V_0$
	\[
		k_2 = \frac{\sqrt{2m(E-V_0)}}{\hbar} =  \frac{\sqrt{ k_1^2 - 2mV_0 }}{\hbar}
	\]
	\item Given the wave functions for regions 1 and 2
	\[
		\Psi_1 = Ae^{ik_1x} + Be^{-ik_1x}
	\]
	\[
		\Psi_2 = Ce^{ik_2x}
	\]
	And the boundary conditions ($x =0 $) 
	\[
		\Psi_1 = \Psi_2
	\]
	\[
		\frac{d}{dx}\Psi_1 = \frac{d}{dx}\Psi_2
	\] 
	Then 
	\[
		A + B = C \text{ and } ik_1(A - B) = ik_2C
	\]
	Solving the system of equations for the reflection coefficient 
	\[
		ik_1(A - B) = ik_2(A+B)
	\]
	\[
		B(ik_1+ik_2) = A(ik_1-ik_2) 
	\]
	\[
		\frac{B}{A} = \frac{k_1-k_2}{k_1+k_2} = \frac{\sqrt{2mE} - \sqrt{2m(E-V_0)}}{\sqrt{2mE} - \sqrt{2m(E-V_0)}}
	\]
	Substituting $E = 2V_0$
	\[
		\frac{B}{A} = \frac{\sqrt{4mV_0} - \sqrt{2mV_0}}{\sqrt{4mV_0} + \sqrt{2mV_0}} = \frac{2-\sqrt{2}}{2 +\sqrt{2}}
	\]
	So
	\[
		R = \left| \frac{B}{A}\right|^2 = \left| \frac{2-\sqrt{2}}{2+\sqrt{2}} \right|^2 \approx 0.02944
	\]
	\item	
	\[
		T = 1 - R = 1- \left| \frac{2-\sqrt{2}}{2+\sqrt{2}} \right|^2 \approx 0.97056 
	\]
	\item If $n$ is the number of transmitted particles
	\[
		n = T(10^6) =  9.7056\e{5}\un{particles}
	\]
	Classically, all of the particles are transmitted.
	
\end{enumerate}

\item [46)]
\begin{enumerate}
    \item Given (from problem 44)
	\[
		k_1 = \frac{\sqrt{2mE}}{\hbar}
	\]
	And $E \Rightarrow E - (-V_0)$
	\[
		k_2 = \frac{\sqrt{2m(E-V_0))}}{\hbar} = \frac{\sqrt{k_1^2 + 2mV_0}}{\hbar}
	\]
	\item Also given (from problem 44)
	\[
		\frac{B}{A} = \frac{k_1-k_2}{k_1+k_2} = \frac{\sqrt{2mE} - \sqrt{2m(E+V_0)}}{\sqrt{2mE} + \sqrt{2m(E+V_0)}}	
	\]
	Substituting $E = 2V_0$
	\[
		\frac{B}{A} = \frac{\sqrt{4mV_0} - \sqrt{4mV_0}}{\sqrt{6mV_0} + \sqrt{6mV_0}} = \frac{2-\sqrt{6}}{2 +\sqrt{6}}
	\]
	So 
	\[
		R =  \left| \frac{2-\sqrt{6}}{2+\sqrt{6}} \right|^2 \approx 0.01021
	\]
	\item	
	\[
		T = 1- \left| \frac{2-\sqrt{6}}{2+\sqrt{6}} \right|^2 \approx 0.98979	
	\]
	\item If $n$ is the number of transmitted particles
	\[
		n = T(10^6) =  9.8979\e{5}\un{particles}	
	\]
\end{enumerate}

\item [51)]
\begin{enumerate}
    \item With
	\[
		k_1 = \frac{\sqrt{2mE}}{\hbar}	
	\]
	And $E \Rightarrow E - (V_0)$
	\[
		k_2 = \frac{\sqrt{k_1^2 - 2mV_0}}{\hbar}
	\]
	Then (again from problem 44 -- I think it is easier than plugging things into the formula formula given in the book)
	\[
		\frac{B}{A} = \frac{k_1-k_2}{k_1+k_2} = \frac{\sqrt{2mE} - \sqrt{2m(E-V_0)}}{\sqrt{2mE} + \sqrt{2m(E-V_0)}}
	\]
	\[
		T = 1 - \left| \frac{B}{A} \right|^2 = 1 - \left| \frac{\sqrt{80} - \sqrt{20}}{\sqrt{80} + \sqrt{20}} \right|^2 = 0.\overline{8}
	\]
	\item The correct answer to this question is given on collab.
\end{enumerate}

\item [53)] The probability of finding some particle particle of wave function $\psi$ on the interval $[a, b]$ is given by 
\[
	\int_a^b \psi^*(x)\psi(x) \; dx
\]
The wavefunction for a particle int he ground state of an infinite square well is given by
\[
	\psi_1(x) = \sqrt{ \frac{2}{L} } \sin \frac{\pi x}{L}	
\]
The probability density integral is then
\[
	\frac{2}{L} \int_a^b \sin^2 \frac{\pi x}{L} \; dx = \frac{1}{L} \left[ x - \frac{L}{2\pi}\sin \frac{2 \pi x}{L} \right]_a^b	
\]
\begin{enumerate}
    \item For $ 0 < x < 1/2 L $\\
	I hold that symmetry is enough to establish that the answer is 0.5. The probability density function is symmetric about the $y$ axis.
	\item
	\[
		\frac{1}{L} \left[ x - \frac{L}{2\pi}\sin \frac{2 \pi x}{L} \right]_0^{1/3L} = \frac{1}{L} \left( \frac{L}{3} - \frac{L}{2\pi}\sin\frac{2\pi}{3} \right) = \frac{1}{3} - \frac{\sqrt{3}}{4\pi}
	\]
	\item
	\[
		\frac{1}{L} \left[ x - \frac{L}{2\pi}\sin \frac{\pi x}{L} \right]_0^{3/4L} = \frac{1}{L} \left( \frac{3L}{4} - \frac{L}{2 \pi}\sin \frac{3 \pi}{4} \right) = \frac{3}{4} - \frac{\sqrt{2}}{4 \pi}
	\]
\end{enumerate}

\item [59)]
	\[
		T  = 16 \cdot \frac{E}{V_0} \left( 1 - \frac{E}{V_0} \right)e^{-2\alpha a}
	\]
	\[
		\alpha  = \frac{\sqrt{ 2m(V_0 - E)}}{\hbar} = \frac{\sqrt{2mc^2(V_0-E)}}{\hbar c} = \frac{\sqrt{2(511\un{keV})(15\un{eV})}}{197.3\un{eVnm}} = 1.984\e{10}\un{m^{-1}}
	\]
\begin{enumerate}
	\item Given a width of $a = 1\un{nm}$
	\[
		T  = 16 \cdot \frac{10}{25} \left( 1 - \frac{10}{25} \right)e^{-2(1 \un{nm})\alpha } = 2.246\e{-17}
	\]
	\item Given a width of $0.1\un{nm}$
	\[
		T  = 16 \cdot \frac{10}{25} \left( 1 - \frac{10}{25} \right)e^{-2 (0.1 \un{nm}) \alpha } = 0.07262
	\]

\end{enumerate}

\item [60)] The time dependent Schrödinger equation is 

\begin{enumerate}
\item
	\[
		- \frac{\hbar^2}{2m}\frac{\partial^2 \Psi(x, t)}{\partial x^2} + V(x,t)\Psi(x,t) = i \hbar \frac{\partial \Psi(x, t)}{\partial(t)}
	\]
	Getting the necessary derivatives of $\Psi_1(x, t)$
	\[
		\Psi_1(x, t) = A\sin(kx - \omega t)
	\]
	\[
		\frac{\partial \Psi_1(x, t)}{\partial x} = Ak\cos(kx - \omega t)
	\]
	\[
		\frac{\partial^2 \Psi_1(x, t)}{\partial x^2} = -Ak^2\sin(kx - \omega t)
	\]
	\[
		\frac{\partial \Psi_1(x, t)}{\partial t} = -A\omega\cos(kx - \omega t)
	\]
	Substituting into the Schrödinger equation
	\[
		- \frac{\hbar^2}{2m} (-Ak^2\sin(kx - \omega t)) + V_0 ( A\sin(kx - \omega t) ) =  - i \hbar (A\omega\cos(kx - \omega t))
	\]
	\[
		\left[\frac{\hbar^2Ak^2}{2m} + V_0 \right]\sin(kx - \omega t) + \left[ A\omega i\hbar\right]\cos(kx - \omega t) = 0
	\]
	Because the resulting equation takes the form 
	\[
		C_1\sin(kx - \omega t) + C_2\cos(kx - \omega t) = 0
	\] 
	where $C_1$ and $C_2$ must be zero, $\Psi_1(x, t) = A\sin(kx - \omega t)$ is not a valid wave function.\\

	Getting the necessary derivatives of $\Psi_2(x, t)$
	\[
		\Psi_2(x, t) = A\cos(kx - \omega t)
	\]
	\[
		\frac{\partial \Psi_2(x, t)}{\partial x} = -Ak\sin(kx - \omega t)
	\]
	\[
		\frac{\partial^2 \Psi_2(x, t)}{\partial x^2} = -Ak^2\cos(kx - \omega t)
	\]
	\[
			\frac{\partial \Psi_2(x, t)}{\partial t} = A\omega\sin(kx - \omega t)
	\]
	Substituting into the Schrödinger equation
	\[
		- \frac{\hbar^2}{2m}( -Ak^2\cos(kx - \omega t) ) + V_0 (A\cos(kx - \omega t)) = -i \hbar ( A\omega\sin(kx - \omega t))
	\]
	\[
		 \left[ \frac{\hbar^2Ak^2}{2m} + V_0 \right] \cos(kx - \omega t)  + [ A \omega i \hbar ]\sin(kx - \omega t) = 0
	\]
	Just like the previous example, $C_1$ and $C_2$ must be zero, and thus $\Psi_2(x,t)$ is not a valid wave function.
	
	\item Doing the same procedure with $\Psi_3(x, t,)$
	\[
		\Psi_3(x, t) = Ae^{i(kx-\omega t)}
	\]
	\[
		\frac{\partial \Psi_3(x, t)}{\partial x} = A(ik)e^{i(kx-\omega t)}
	\]
	\[
		\frac{\partial^2 \Psi_3(x, t)}{\partial x^2} = A(ik)^2e^{i(kx -\omega t)}
	\]
	\[
			\frac{\partial \Psi_3(x, t)}{\partial t} = A(-i\omega)e^{i(kx - \omega t)}
	\]
	Substituting into the Schrödinger equation
	\[
			- \frac{\hbar^2}{2m} \left( A(ik)^2e^{i(kx -\omega t)} \right)	+ V_0 \left(  Ae^{i(kx-\omega t)} \right) = i\hbar \left(  A(-i\omega)e^{i(kx - \omega t)} \right)
	\]
	\[
		\frac{\hbar^2}{2m} + V_0 = \hbar \omega
	\]
	As the two sides of the equation can be equal without being zero $\Psi_3(x, t)$ satisfies the Schrödinger time dependent wave function.

\end{enumerate}
\item [64)] 
\begin{enumerate}
    \item The general solutions for regions 1,2, and 3 are listed int he textbook (I will not list them here. The corresponding boundary conditions are 
	\[
		\Psi_1(x) = \Psi_2(x) \text{ when } x=0	
	\]
	\[
		\Psi_2(x) = \Psi_3(x) \text{ when } x=a	
	\]
	\[
		\frac{\partial \Psi_1(x)}{\partial x} = \frac{\partial \Psi_2(x)}{\partial x} \text{ when } x= 0
	\] 
	\[
		\frac{\partial \Psi_2(x)}{\partial x} = \frac{\partial \Psi_3(x)}{\partial x} \text{ when } x=a
	\]
	Using the boundary conditions, simultaneous equations can be created for the coefficients.
	\[
		A+B = C+D	
	\]
	\[
		ik_1A - ik_1B = \alpha D - \alpha C	
	\]
	\[
		Ce^{-\alpha a} + De^{\alpha a} = Fe^{ik_1 a}
	\]
	\[
		\alpha D e^{\alpha a} - \alpha C e^{-\alpha} = ik_1 Fe^{ik_1 a}
	\]
	I did not solve the system of equations by hand (I used Mathematica), but the desire result is the following equation for A in terms of F
	\[
		|A|^2 = \frac{|F|^2}{4} \left[ 4 + \left( \frac{V_0^2}{E(V_0-E)} \right)\sinh^2 \alpha a \right]
	\]
	Solving for $\frac{|F|^2}{|A|^2} = $
	\[
		\frac{|F|^2}{|A|^2} = T = \left[ 1 - \frac{\sinh^2 \alpha a}{\left( \frac{4E}{V_0} \right) \left( 1 - \frac{E}{V_0} \right)} \right]^{-1}	
	\]

	\item Substituting 
		\[
			\sinh \alpha a = \frac{e^{\alpha a} - e^{-\alpha a}}{2}
		\]
		As $a\alpha >> 1$ the equation tends to 
		\[
			T = \left[ \frac{e^{2a\alpha}}{16 \left( \frac{E}{V_0} \right) \left( 1 - \frac{E}{V_0}\right)}  \right]^{-1}
		\]
		or 
		\[
			T = 16 \left( \frac{E}{V_0} \right) \left( 1 - \frac{E}{V_0} \right)e^{-2 a \alpha}
		\]
\end{enumerate}

\item [65)] For $x > 0$, the wave function is
\[
	\Psi(x) = Ce^{-\frac{\sqrt{2m(V_0-E)}}{\hbar}x}
\]
And thus the probability density is given by 
\[
	| \Psi |^2 = |C|^2e^{-2\frac{\sqrt{2m(V_0-E)}}{\hbar}x}
\]
If $|C|$ is given by 
\[
	\left| \frac{2\sqrt{E}}{\sqrt{E} + \sqrt{E - V_0}} \right|^2 |A|^2= 	\left| \frac{2\sqrt{20\un{eV}}}{\sqrt{20\un{eV}} + \sqrt{20\un{eV} - 40\un{eV}}} \right|^2 = \left|\frac{2}{1+i} \right|^2 = \frac{4}{2} = 2
\]
And
\[
		\alpha  = \frac{\sqrt{ 2m(V_0 - E)}}{\hbar} = \frac{\sqrt{2mc^2(V_0-E)}}{\hbar c} = \frac{\sqrt{2(511\un{keV})(20\un{eV})}}{197.3\un{eVnm}} = 2.29\e{10}\un{m^{-1}}
\]
The graph able equation is then
\[
	|\Psi|^2  = 2e^{-2x(2.29\e{10}\un{m^{-1}})}
\]

\begin{figure}[h!]
\begin{center}
   \fbox{\includegraphics[width=120mm]{plot.png}}
\end{center}
\end{figure}

\end{enumerate}
\end{document}
