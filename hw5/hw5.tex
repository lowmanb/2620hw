\documentclass[]{2620hw}

\hwnum{5}
\date{February 21st, 2014}

\begin{document}
\maketitle

\begin{enumerate}

\item [4-19]
\begin{enumerate}
	\item Adjusting for the mass correction,
	\[
		\mu = \frac{m}{1+ \frac{m}{M}}
	\]
	where $M_h = 938.27 \un{MeV/c^2} $
	\[
		\mu = \frac{105.7\un{MeV/c^2}}{1 + \left( \dfrac{105.7\un{MeV/c^2}}{938.27\un{MeV/c^2}} \right) } = 94.998\un{MeV/c^2} 
	\]
    \item Given the Bhor Radius
	\[
		a_0 = \frac{\hbar}{mc\alpha}	
	\]
	For the muonic atom
	\[ 
		a_0	= \frac{\hbar}{(94.998\un{MeV/c^2})c \alpha} = 285.65\un{fm}
	\]

	\item The energy (magnitude) of the muon in Bhor orbital 1 (the lowest orbital) is given by 
	\[
		E_0 = \frac{mk^2e^4}{2\hbar}	
	\]
	\[
		E_0 = \frac{(94.998\un{MeV/c^2})k^2e^4}{2 \hbar} = 2.529\un{keV} 
	\]
	\item Calculating the Rydberg Constant 
	\[
		R = \frac{mk^2e^4}{4 \pi c \hbar^3} = \frac{(94.998\un{MeV/c^2})k^2e^4}{4 \pi c \hbar^3} = 2.0401 \e{9}\un{m}^{-1}
	\]
	The lowest wavelength for the Lyman series is given by the Atomic Spectra equation
	\[
		\frac{1}{\lambda} = R \left( \frac{1}{1^2} - \frac{1}{\infty} \right)
	\]
	\[
		\lambda = \frac{1}{R} = \frac{1}{2.0401\e{9}\un{m}^{-1}} = 0.490 \un{nm}
	\]

\end{enumerate} 

\item [4-22]
\begin{enumerate}
    \item 
	\[
		E = - \text{ Rydberg Energy } = -E_r  = 13.6 \un{keV} 
	\]
	\item If the total momentum is zero, the momentum of atom is the same in magnitude as the momentum of the photon. The momentum of the photon is given by
	\[
		p = \frac{E_r}{c}
	\]
	So the (non relativistic) kinetic energy of the atom (the excitation energy) is
	\[
		E_k =  \frac{1}{2}mv^2 = \frac{1}{2}{m}\left( \frac{P}{m} \right)^2 = \frac{P^2}{2m} = \frac{E_r^2}{2mc^2} 
	\]
	The excitement energy stored in the $n=2$ orbital is 
	\[
		E_2 = E_r \left( \frac{1}{1^2} - \frac{1}{2^2} \right) = \frac{3}{4}E_r
	\]
	The proportion of energy carried in the second orbital to the excitement energy is then 
	\[
		\frac{E_2}{E_k} = \frac{3mc^2}{2E_r}
	\]
\end{enumerate}

\item [4-58] The velocity of $Z$ electrons orbiting at radius $r$ from a nucleus is given by solving the following relation:
\[
	\frac{mv^2}{r} = \frac{GZMm}{r^2}
\]
\[
	v = \sqrt{ \frac{GZM}{r} }
\]
Using Bhors quantization of angular momentum
\[
	L = mvr = n \hbar
\]
\[
	r = \frac{n \hbar}{m} \left( \frac{r}{GZm} \right)^{1/2} = \frac{n^2}{Z}a_0
\]
Where
\[
	a_0 = \frac{\hbar^2}{GMm^2} = 12.69 \un{\; trillion\; light\; years}
\]
The total energy of the electron is the sum of the kinetic and potential energy 
\[
	E = GZMm  \left( \frac{1}{2r} - \frac{1}{r} \right)= - \frac{GZMm}{2r}
\]
Substituting $r$
\[
	E_n = - \frac{GZMm}{2} \left( \frac{Z}{n^2 a_0} \right)
\]
\[
	E_n = -E_0 \frac{Z^2}{n^2}
\]
Where 
\[
	E_0 = \frac{GMm}{2a_0} = 2.643\e{-78}\un{eV}
\]
Following Bhor's second postulate
\[
	hf = E_{n_i} - E_{n_f} = E_0Z^2 \left( \frac{1}{n_f^2} - \frac{1}{n_i^2} \right)
\]
The energy of the $H_{\alpha}$ line is then
\[
	E = 2.643\e{-78}\un{eV} \left( \frac{1}{2^2} - \frac{1}{3^2} \right) = 3.671\e{-79}\un{eV}
\]
And the frequency 
\[
	f = \frac{E}{h} = \frac{3.671\e{-79}\un{eV}}{h} = 8.876\e{-65}\un{Hz}
\]
Finally, the limit of the Balmar series is
\[
	\lim_{n_i \rightarrow \infty} E_0\left( \frac{1}{2^2} - \frac{1}{n_i^2} \right) = \frac{E_0}{4} =  \frac{2.643\e{-78}\un{eV}}{4} = 6.608\e{-79}\un{eV}
\]

As $a_0$ and $E_0$ for ``real" hydrogen are given in the text, I will not compute them here.\\\\
It is apparent that the values computed for a Hydrogen atom with no electrostatic charge are wildly different from the actual values. The calculated $a_0$ radius is \textit{much} larger than the actual $a_0$ and is on the order of 100 times the size of the observable universe. Conversely, the computed $E_0$ is \textit{much} smaller than the actual $E_0$.

Without additional computation, then, one can conclude (via the relative change in $E_0$) that the computed frequency and energy of the $H_{\alpha}$ line as well are the limit of the Balmar series are \textit{much} smaller than the actual values. 

\item [5-4] From 4-22 (b)
	\[
		E_k = \frac{p^2}{2m} 
	\]
	Therefore 
	\[
		p = \sqrt{2E_km}
	\]
	Given also the DeBrogile relation
	\[
		\lambda = \frac{h}{p}
	\]
	The wavelength in terms of the (non relativistic) kinetic energy of a particle is
	\[
		\lambda = \frac{h}{\sqrt{2E_km}}
	\]
	$E_k = 4.5\un{keV}$ in all cases
\begin{enumerate}
    \item For an electron, $m = 510.999\un{keV/c^2}$
	\[
		\lambda = \frac{h}{\sqrt{2(4.5\un{keV})(510.999\un{keV/c^2})}} = 18.28\un{pm}
	\]
	\item For a proton, $m = 938.27\un{keV/c^2}$
	\[
		\lambda = \frac{h}{\sqrt{2(4.5\un{keV})(938.27\un{keV/c^2})}} = 426.7\un{fm}
	\]
	\item For an alpha particle, $m = 3.728\un{GeV/c^2})$
	\[
		\lambda = \frac{h}{\sqrt{2(4.5\un{keV})(3.728\un{GeV/c^2})}} = 214.1\un{fm}
	\]
	
	
\end{enumerate}

\end{enumerate}

\end{document}
