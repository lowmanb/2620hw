\documentclass{2620hw}

\hwnum{13}
\due{April 25, 2014}

\begin{document}
\maketitle

\begin{enumerate}

\item [11-100] Applying conservation of energy
\[
	17.7 \un{MeV} = K_{He} + K_n = \frac{1}{2}m_{He}v_{He}^2 + \frac{1}{2}m_{n}v_{n}^2
\]
And applying conservation of momentum
\[
	m_{He}v_{He} + 	m_{n}v_{n} = 0
\]
Solving for $v_{He}$
\[
	v_{He} = -\frac{m_nv_n}{m_{He}}
\]
And substituting in the first equation
\[
	17.7\un{MeV} = \frac{1}{2}m_{He} \left( \frac{m_n}{m_{He}}\right)^2 v_n^2 + \frac{1}{2}m_{n}v_{n}^2
\]
\[
	= \frac{1}{2}m_{n}v_n^2 \left(1 +  \frac{m_n}{m_{He}} \right)
\]
\[
	= K_n \left(1 +  \frac{m_n}{m_{He}} \right)
\]
Solving for $K_n$
\[
	K_n = \frac{17.7\un{MeV}}{1+ \cfrac{m_n}{m_{He}}} = \frac{17.7\un{MeV}}{1+ \cfrac{1.008665\un{u}}{4.002603\un{u}}} = 11.77 \un{MeV}
\]
And solving for $K_{He}$
\[
	K_{He} = 17.7\un{MeV} - K_n = 17.7\un{MeV} - 11.77\un{MeV} =  5.93\un{MeV}
\]
\newpage

\item [12-9]
\begin{enumerate}
	\item Weak Interaction
	\item Electromagnetic Interaction
	\item Strong Interaction
	\item Weak Interaction
\end{enumerate}

\item [12-10] The latter reaction is the longest. The first reaction, because of the presence of photons, is an Electromagnetic reaction. The second reaction cannot be a strong interaction due to the lepton products, and the presence of neutrinos and anti neutrinos ensures that the reaction will be weak. The textbook establishes that weak interactions are always slower than Electromagnetic interactions.

\item [12-14] The decay modes for $\Sigma^+$ and $\Sigma^-$ are different
\[
	\Sigma^+ \rightarrow n + \pi^{\circ}
\]
\[
	\Sigma^- \rightarrow n + \pi^{-}
\]
The products of the reactions have different masses, and therefore $\Sigma^+$ and $\Sigma^-$ must have different masses.

\item [12-17]
\begin{enumerate}
	\item Conservation of energy is violated ($m_p < m_n$)
	\item Conservation of energy is violated ($m_n < m_p + m_{\pi}$)
	\item Conservation of momentum is violated. At least two $\gamma$ rays must be emitted to conserve momentum.
	\item OK
	\item OK
\end{enumerate}

\item [12-19]
\begin{enumerate}
	\item Strangness of:
	\begin{itemize}
		\item $\Omega^- = -3$ 
		\item $\equiv^{\circ} = -2$ 
		\item $\pi^{-} = 0$ 
	\end{itemize}
	$\delta S = -2 - (-3) = 1$\\
	Therefore goes via weak interaction.

	\item Strangness of:
	\begin{itemize}
		\item $\equiv^{\circ} = -2$ 
		\item $P = 0$ 
		\item $\pi^{-} = 0$ 
		\item $\pi^{\circ} = 0$ 
		\item $\Omega^- = -3$ 
	\end{itemize}
	$\delta S = 0 - (-2) = 2$\\
	Therefore does not go under any interaction

	\item Strangness of:
	\begin{itemize}
		\item $\wedge^{\circ} = -1$ 
		\item $P = 0$ 
		\item $\pi^{-} = 0$ 
	\end{itemize}
	$\delta S = 0 - (-1) = 1$\\
	Therefore goes via weak interaction.
\end{enumerate}

\item [12-21]
\begin{enumerate}
	\item $n+n$\\
	\[
		T \in \{-1, 0, 1\}
	\]
	Since $T_3 = -\frac{1}{2}$ for the neutron (as proton is $\frac{1}{2}$),
	\[
		T_{3_{net}} = -\frac{1}{2}-\frac{1}{2} = -1
	\]
	
	\item $n+p$
	\[
		T \in \{-1, 0, 1\}
	\]
	Since $T_3 = -\frac{1}{2}$ for the neutron and $\frac{1}{2}$ for the proton,
	\[
		T_{3_{net}} = \frac{1}{2}-\frac{1}{2} = 0
	\]

	\item $\pi^+ +p$
	\[
		T \in \{-\frac{3}{2}, \frac{3}{2}\}
	\]
	Since $T_3 = \frac{1}{2}$ for proton and 1 for $\pi^+$,
	\[
		T_{3_{net}} = \frac{1}{2}+1 = \frac{3}{2}
	\]

	\item $\pi^- +n$
	\[
		T \in \{-\frac{3}{2}, \frac{3}{2}\}
	\]
	Since $T_3 = -\frac{1}{2}$ for neutron and -1 for $\pi^-$,
	\[
		T_{3_{net}} = -\frac{1}{2}-1 = -\frac{3}{2}
	\]

	\item $\pi^+ +n$
	\[
		T \in \{-\frac{3}{2},-\frac{1}{2}, \frac{1}{2}, \frac{3}{2}\}
	\]
	Since $T_3 = -\frac{1}{2}$ for neutron and 1 for $\pi^-$,
	\[
		T_{3_{net}} = -\frac{1}{2}+1 = 1
	\]
\end{enumerate}

\item [12-24] As $K^-$, $K^+$, $K^0$ and $\pi^- = 0$ and $p = 1$,
\[
	0 + 1 \rightarrow 0 + 0 + \Omega^- 
\]
\[
	\Omega^- = 1
\]
Plugging into second reaction
\[
	1 \rightarrow \; \equiv^{\circ} + \; 0
\]
\[
	\equiv^{\circ} \; = 1
\]

\end{enumerate}

\end{document}
