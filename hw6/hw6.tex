\documentclass[]{2620hw}

\hwnum{6}
\due{February 28th, 2014}

\begin{document}
\maketitle
\begin{enumerate}

\item [5-13]
	The spacing of the Bragg planes $d$ is related to the spacing of the atoms $D$ by
	\[
		d = D \sin{\alpha}
	\]
	Solving for $D$
	\[
		D = \frac{d}{\sin{\alpha}}
	\]
	In terms of $\lambda$, the equation becomes
	\[
		n\lambda = D \sin{\varphi}
	\] 
	or
	\[
		 \lambda = \frac{d\sin{\varphi}}{n \sin{\alpha}} =   \frac{(0.3\un{nm}) (\sin{42})}{\sin{21}} = 0.560 \un{nm}
	\]
	Given the DeBrogile relation 
	\[
		\lambda = \frac{h}{p} =  \frac{h}{\sqrt{ 2mE_k }}
	\]
	Solving for $E_k$
	\[
		E_k = \frac{h^2}{2m\lambda^2} = \frac{h^2}{2(938.27 \un{MeV/c^2})(0.560\un{nm})^2} = 2.61 \un{meV}	
	\]

\item [5-21]
	Given the uncertainty relation \footnote{Any equations involving unceratainty are notated with an equals sign. If solving for one uncertainty given the other, it can be assumed that the result is the mimumum value.} 
	\[
		\Delta \omega \Delta t = 2\pi \Delta f \Delta t = 1
	\]
	Solving for $\Delta t$
	\[
		\Delta t = \frac{1}{2\pi \Delta f} = \frac{1}{2\pi (5000 \un{Hz})} = 31.83 \un{\mu s}
	\]

\item [5-34]
	The Heisenburg uncertainty is given by 
	\[
		\Delta E \Delta t = \frac{\hbar}{2}
	\]
	Solving for $\Delta t$
	\[
		\Delta t = \frac{\hbar}{2 \Delta E}
	\]
	And susbstituting $\Delta E = 2\pi\hbar \Delta f$
	\[
		\Delta t = \frac{1}{4\pi \Delta f}
	\]
	With $\Delta f = 3.5 \e{14}\un{Hz}$
	\[
			\Delta t = \frac{1}{4\pi ( 3.5 \e{14}\un{Hz})} = 0.227 \un{fs}
	\]

\item [5-35]
	Given the DeBrogile Relation
	\[
		\lambda = \frac{h}{p} = \frac{h}{\sqrt {2mE_k}} = \frac{h}{\sqrt{2(938.27\un{MeV/c^2}(10\un{MeV})}} = 9.05 \un{fm}
	\]
	To observe neutron wave behavior, then, the neutron would have to interact with a particle smaller that $9.05 \un{fm}$. It seems (from a quick Google query) that the diameter of an electron in smaller than this distance. 

\item [5-42]
\begin{enumerate}
	\item [(b)] For a ``Particle in a Box"
	\[
		E_k = \frac{\hbar^2}{2mL^2} = \frac{\hbar^2}{2(938.27 \un{MeV/c^2})(10^{-15}\un{m})^2} = 20.75 \un{MeV}
	\]
	\item [(a)]
	The velocity of the proton is then
	\[
		v = \sqrt{ \left(\frac{2E_k}{m}\right) } = \sqrt{ \left(\frac{20.75 \un{MeV}}{938.27 \un{MeV/c^2}}\right)} = 0.21{c}
	\]
	\item [(c)] For an electron
	\[
		E_k = \frac{\hbar^2}{2(511\un{keV/c^2})(10^{-15}\un{m})} = 38.1 \un{GeV}
	\]
\end{enumerate}

\item [5-51]
\begin{enumerate}
    \item Applying the DeBrogile relation for both the electron and positron
	\[
		\lambda = \frac{h}{p} = \frac{h}{mv} = \frac{h}{(511\un{keV})(3\e{6}\un{m/s})} = 2.42 \un{nm}
	\]
	\item The energy of the system is given by (classical kinetic energy is negligible) 
	\[
		2mc^2 = 2E_{\text{photon}}
	\]
	\[
		E_{\text{photon}} = mc^2 = (511\un{keV}/c^2)c^2 = 511\un{keV}
	\]
	\item The momentum of each photon is then 
	\[
		p = \frac{E}{c} = 511\un{keV/c}
	\]
	\item And the wavelength
	\[
		\lambda = \frac{h}{p} = \frac{h}{511\un{keV/c}} = 2.43 \un{pm}
	\]
	
\end{enumerate}

\end{enumerate}

\end{document}
