\documentclass[]{2620hw}

\hwnum{7}
\due{March 7th, 2014}

\begin{document}
\maketitle
\begin{enumerate}
    
\item [5-52] 
\begin{enumerate}
	\item 
	\[
		E = p = n + \pi^+
	\]
	Taking the difference of sides of the equation
	\[
		(n + \pi^2) - p = 140 \un{MeV}	
	\]
	\item From the Heisenberg uncertainty principle
	\[
		\Delta E \Delta t \sim \frac{\hbar}{2}
	\]
	Solving for t
	\[
		\Delta t = \frac{\hbar}{2\Delta E} =  \frac{\hbar}{2(140 \un{MeV})} = 2.35 \e{-24}\un{s}
	\]
	\item 
	\[
		x = \frac{\Delta t}{c} = \frac{2.35 \e{-24}\un{s}}{c} = 7.05 \e{-16}\un{m}
	\]
	\item 
	\[
		n = \frac{1\un{\mu s}}{2.35 \e{-24}\un{s}} = 4.26 \e{17}
	\]
	
\end{enumerate}
	
\item [6-14]
\begin{enumerate}
	\item Given Heisenberg's uncertainty principle
	\[
		\Delta x \Delta p \sim \hbar
	\]
	Solving for $\Delta p$
	\[
		\Delta p \sim \frac{\hbar}{\Delta x} \sim \frac{\hbar}{L}
	\]
	The energy of the particle is given by the equation
	\[
		E = \frac{p^2}{2m}
	\]
	So the smallest possible energy is given by 
	\[
		E = \frac{\hbar^2}{2mL^2}
	\]
	\item Given equation 6-24
	\[
		E_n = n^2 \frac{\pi^2 \hbar^2}{2mL^2}
	\]
	For $n=1$
	\[
		E_1 = \frac{\pi^2 \hbar^2}{2mL^2}
	\]
	The minumum allowed energy according to the time-independent Schrodinger equation is greater than the minimum allowed energy according to the Heisenberg uncertainty equation by a factor of $\pi^2$.
\end{enumerate}

\item [6-21]
	Given equation 6-24
	\[
		E_n = n^2 \frac{\pi^2 \hbar^2}{2mL^2}
	\]
\begin{enumerate}
	\item For electron $n-1$
	\[
		E_1 = \frac{\pi^2 \hbar^2}{2(511\un{keV})(10\un{fm})} = 3.76 \un{GeV}
	\]
	\item 
	\[
		E_1 = \frac{\pi^2 \hbar^2}{2(938\un{MeV})(10\un{fm})} = 2.05 \un{MeV}
	\]
	\item As the second state for each particle is simply a factor of four larger, the difference between the first and second state in each case is three times the energy of the particle in state 1.
	
\end{enumerate}

\item [6-22]
	Given equation 6-24
	\[
		E_n = n^2 \frac{\pi^2 \hbar^2}{2mL^2}
	\]
	And 
	\[
		F = \frac{-dE_n}{dL}
	\]
	Then
	\[
		F = -2 n^2 \frac{\pi^2 \hbar^2}{2mL^3} = -\frac{\pi^2 \hbar^2}{(511\un{keV})(10\e{-10}\un{m})^3} = -1.2 \e{-7}\un{N}
	\]
	The weight of an electron is given by its mass times the acceleration of gravity
	\[
		F = mg = (511\un{keV})(9.91\un{m/s^2}) = 8.93\e{-30}\un{N} 
	\]
	This result is much smaller than the quantum mechanical result. **Still need to answer why

\item [6-23]
The wave function of a particle in some state $a$ inside an infinite square well is given by 
\[
	\Psi_a(x) =\sqrt{\frac{2}{L}} \sin{\left( \frac{a\pi x}{L} \right)}
\]
Evaluating the integral
\[
	0 = \int_0^L \Psi_n(x) \Psi_m(x) \; dx
\]
\[
	= \int_0^L \sqrt{\frac{2}{L}} \sin{\left( \frac{n\pi x}{L} \right)} \sqrt{\frac{2}{L}} \sin{\left( \frac{m\pi x}{L} \right)} \; dx
\]
\[
	= \frac{1}{L}\int_0^L 2\sin{\left( \frac{n\pi}{L}x \right)}\sin{\left( \frac{m\pi}{L}x \right)} \; dx
\]
\[
	= \frac{1}{L} \int_0^1 \cos{ \left( \frac{m+n}{L} \pi x \right) } - \cos{ \left( \frac{m-n}{L} \pi x \right) } \; dx
\]
\[
	= \frac{1}{L}\left( \frac{\sin{\left( \frac{m+n}{L} \pi L \right)}}{\left( \frac{m+n}{L} \pi \right)} - \frac{\sin{\left( \frac{m-n}{L} \pi L \right)}}{\left( \frac{m-n}{L} \pi \right)} \right)
\]
Clearing denominators
\[
	0 = \sin{((m+n)\pi)} - \sin{((m-n)\pi)}
\]
\[
	\sin{((m+n)\pi)} = \sin{((m-n)\pi)}
\]
In this form it is apparent that $m$ and $n$ are orthogonal.

\end{enumerate}
\end{document}


