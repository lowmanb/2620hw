\documentclass[12pt]{2620hw}

\hwnum{11}
\due{April 11, 2014}

\begin{document}
\maketitle

\begin{enumerate}

\item [7-31] Given that the magnitude of the angular momentum is
\[
	|S| = (3/4)^{1/2} \hbar
\]
The rotational velocity of the particle is then 
\[
	v_r = \frac{|S|}{m_er} = \frac{(3/4)^{1/2}\hbar}{(511\un{keV/c^2})(10^{-15}\un{m})} \approx 1\e{11}\un{m/s} 
\]
or 3 orders of magnitude greater thatn the speed of light -- which is impossible.

\item [7-37] 
\begin{enumerate}
	\item Given the possible values of $j$
	\[
		j = | \ell + s | \;\;\text{or}\;\; j = | \ell - s | 
	\]
	Then 
	\[
		j = 2 + \frac{1}{2} = \frac{5}{2} \;\;\text{or}\;\; j = 2 - \frac{1}{2} = \frac{3}{2}
	\]
	\item The magnitude of the total anuglar momentum is given by
	\[
		|J|	= \sqrt{ j ( j + 1) } \hbar
	\]
	\begin{itemize}
		\item For $j = \frac{5}{2}$
		\[
			|J|	= \sqrt{ \frac{5}{2} \left( \frac{5}{2} + 1 \right) } \hbar  = \sqrt{ \frac{35}{4} } \hbar
		\]
		\item For $j = \frac{3}{2}$
		\[
			|J|	= \sqrt{ \frac{3}{2} \left( \frac{3}{2} + 1 \right) } \hbar = \sqrt{ \frac{15}{4} } \hbar
		\]
		
	\end{itemize}
	\item The $z$ component of the angular momentum is given by 
	\[
		J_z = m_j \hbar	
	\]
	\begin{itemize}
		\item For $j = \frac{3}{2}$
		\[
			m_j = \pm \frac{1}{2}, \pm \frac{3}{2}
		\]
		Therefore
		\[
			J_z = \pm \frac{1}{2}\hbar, \pm \frac{3}{2}\hbar
		\]
		\item For $j = \frac{5}{2}$
		\[
			m_j = \pm \frac{1}{2}, \pm \frac{3}{2}, \pm \frac{5}{2}
		\]
		Therefore
		\[
			J_z = \pm \frac{1}{2}\hbar, \pm \frac{3}{2}\hbar, \pm \frac{5}{2}\hbar
		\]
		
	\end{itemize}
\end{enumerate}

\item [7-40] For $\ell_1 = \ell_2 = 1$ and $s_1 = s_2 = \frac{1}{2}$
\begin{enumerate}
	\item Given
	\[
		L \in \{\ell_1 + \ell_2, \ell_1 + \ell_2 -1, ..., |\ell_1 - \ell_2 | \}
	\]
	It follows that 
	\[
		L = 0,1,2
	\]
	\item Given
	\[
		S \in \{s_1 + s_2, s_1 + s_2 -1, ..., |s_1 - s_2 | \}
	\]
	It follows that
	\[
		S = 1, 0
	\]
	\item Given 
	\[
		J = L + S
	\]
	Finding all combinations of values from (a) and (b) is trivial
	\[
		J = 0,1,2,3	
	\]
	\item Given 
	\[
		j_1 \in \{\ell_1 + s_1, \ell_1 + s_1 -1, ..., |\ell_1 - s_1 | \}
	\]
	It follows that 
	\[
		j_1 =  \frac{3}{2}, \frac{1}{2} \;\; \text{and} \;\; j_2 =  \frac{3}{2}, \frac{1}{2}
	\]
	\item Given that 
	\[
		J = j_1 + j_2
	\]
	It is again trivial to find all combinations of $j_1$ and $j_2$ from part (d)
	\[
		J = 0,1,2,3
	\]
	This result does indeed match that from part (c)
\end{enumerate}

\item [7-45] Given the energy of an infinite square well in the $n$th state
\[
	E_n = \frac{n^2 \hbar^2 \pi^2}{2mL^2}
\]
\begin{enumerate}
	\item Given that there are 5 electrons in the well, the energy level distribution must be (by Pauli exclusion)
	\[
		E_{ \text{total} } = 2E_1 + 2E_2 + E_3
	\]
	So the total energy is 
	\[
		E_{ \text{total} } = \frac{\hbar^2 \pi^2}{2mL^2} \left( 2(1^2) + 2(2^2) + 3^2 \right) =  \frac{19\hbar^2 \pi^2}{2(511\un{keV/c^2})(1 \un{nm})^2}= 7.14\un{eV} 
	\]
	\item Becuase pions do not obey the Pauli exclusion principle, all five can stay in the $n=1$ state
	\[
		E = \frac{\hbar^2 \pi^2}{2mL^2} \left( 5(1^2) \right) =\frac{5\hbar^2 \pi^2}{2(264)(511\un{keV/c^2})(1 \un{nm})^2}= 7.11\un{meV} 
	\]
\end{enumerate}

\item [7-50] Given that 
\[
	E = \frac{Z^2(13.6\un{eV})}{n^2} 
\]
Then $Z_{ \text{eff}}$ for the penetrating 3s state is 
\[
	Z_{ \text{eff} } = \sqrt{ \frac{n^2E}{13.6\un{eV}} } = \sqrt{ \frac{9(5.14\un{eV})}{13.6\un{eV}} } = 1.84
\]

\item [7-68] Using basic coordinate identities, we know that
\[
	\cos \theta = \frac{L_z}{L}
\]
It is also known that 
\[
	L_z = m_{j}\hbar
\]
and 
\[
	L = \sqrt{ \ell (\ell +1) } \hbar
\]
Substituting
\[
	\cos \theta = \frac{m_{j}\hbar}{\sqrt{ \ell (\ell +1) } \hbar}  = \frac{m_{j}}{\sqrt{ \ell (\ell +1) }}
\]
Substituting $\ell$ for $m_{j}$ and solving for $\theta_{\text{min}}$
\[
	\theta_{\text{min}} = \cos^{-1} \left(\frac{\ell}{\sqrt{\ell(\ell+1)}} \right)
\]
Using the trigonometric identity
\[
	\sin^2 \theta = 1 - \cos^2 \theta
\]
\[
	\sin^2 \theta = 1 - \left( \frac{\ell}{\sqrt{ \ell (\ell +1) }}\right)^2 = 1 - \frac{\ell^2}{\ell(\ell+1)}
\]
\[
	= \frac{\ell(\ell+1)- \ell^2}{\ell(\ell+1)} = \frac{1}{\ell + 1}
\]
As $\ell \rightarrow \infty$
\[
	\theta_{\text{min}} = \frac{1}{\sqrt{\ell}}
\]
By the small angle approximation.

\item [7-70] Given that the wave function for a hydrogen atom in its ground state is 
\[
	\Psi(r) = \frac{1}{\sqrt{\pi a_0^3}}e^{\frac{-r}{a_0}}
\]
And the radial probability density is just the probability density $|\Psi(r)$ squared integrated over the surface area of the sphere (this was in the last problem set)
\[
	P(R) = \int_0^R \frac{4 \pi r^2}{\pi a_0^3}e^{\frac{-2r}{a_0}} \; dr  = \frac{4}{a_0^3}\int_0^Rr^2 e^{\frac{-2r}{a_0}} \; dr 
\]
Since $e^{\frac{-2r}{a_0}} = 1$
\[
	P(R) = \frac{4}{a_0^3}\int_0^Rr^2 \; dr = \frac{4}{3}\frac{R^3}{a_0^3}
\]
Substituting known values
\[
	P(R) = \frac{4}{3}\frac{(10^{-15}\un{m})^3}{(0.0529\un{nm})^3}	 = 9\e{-15}
\]

\end{enumerate}

\end{document}
